\documentclass[usenatbib,onecolumn]{mnras}
\pdfoutput=1
\pdfminorversion=5

%\usepackage{amsmath}
\usepackage{mathtext,amssymb,amsmath}
\usepackage{epsfig}
\usepackage{graphics}
\usepackage{url}

\usepackage{times}
%\usepackage{helvet}
\usepackage[T1]{fontenc} 
\usepackage{aecompl}

%%%%% AUTHORS - PLACE YOUR OWN MACROS HERE %%%%%

\newcommand{\alert}[1]{\color{red} #1\color{black}}

\renewcommand{\vector}[1]{\ensuremath{\mathbf{#1}}}
\renewcommand{\div}{\ensuremath{-}}

\newcommand{\bea}{\begin{eqnarray}}
\newcommand{\eea}{\end{eqnarray}}
  
\newcommand{\Mach}{\ensuremath{\mathcal{M}}}
\newcommand{\rot}{\ensuremath{\mathbf{curl\,}}}
\newcommand{\mdot}{\ensuremath{\dot{m}}}
\newcommand{\Msun}{\ensuremath{\,\rm M_\odot}}
\newcommand{\Msunyr}{\ensuremath{\,\rm M_\odot\, \rm yr^{-1}}}
\newcommand{\ergl}{\ensuremath{\,\rm erg\, s^{-1}}}
\newcommand{\Gyr}{\ensuremath{\,\rm Gyr}}
\newcommand{\yr}{\ensuremath{\,\rm yr}}
\newcommand{\mum}{\ensuremath{\,\rm \mu m}}
\newcommand{\pc}{\ensuremath{\,\rm pc}}
\newcommand{\cmc}{\ensuremath{\,\rm cm^{-3}}}
\newcommand{\cmsq}{\ensuremath{\,\rm cm^{-2}}}

\newcommand{\Ry}{\ensuremath{\,{\rm Ry}}}
\newcommand{\eV}{\ensuremath{\,{\rm eV}}}
\newcommand{\keV}{\ensuremath{\,{\rm keV}}}
\newcommand{\GeV}{\ensuremath{\,{\rm GeV}}}
\newcommand{\TeV}{\ensuremath{\,{\rm TeV}}}
\newcommand{\AAA}{\ensuremath{\,{\rm \AA}}}
% \newcommand{\ion}[2]{ \ensuremath{ \rm #1\,\sc #2 }}

\newcommand{\gfun}[1]{\ensuremath{{\rm G}\,(#1)}}
\newcommand{\acos}{\ensuremath{\rm acos}\,}
\newcommand{\litwo}{\ensuremath{\rm Li}_2\,}
\newcommand{\lithree}{\ensuremath{\rm Li}_3\,}
\newcommand{\li}[2]{{\rm Li}_{#1}\!\left(#2\right)}
\newcommand{\gf}{\ensuremath{\frac{\sqrt{g_{\varphi\varphi}}}{\alpha}}}
\newcommand{\pardir}[2]{\ensuremath{\frac{\partial #2}{\partial #1} }}
\newcommand{\pardirsec}[2]{\ensuremath{\frac{\partial^2 #2}{\partial #1^2} }}
\newcommand{\ppardir}[2]{\ensuremath{\frac{\partial }{\partial #1} \left( #2\right)}}
\newcommand{\eps}{\epsilon}
\newcommand{\cs}{\ensuremath{c_{\rm s}}}
\renewcommand{\i}{\ensuremath{{\rm i}}}
%\newcommand{\tcfour}{3C\,454.3}
%\newcommand{\tctwo}{3C\,279}
%\newcommand{\pkst}{PKS 1222$+$21}
%\newcommand{\pksf}{PKS 1510$-$89}
% \newcommand{\grs}{GRS~1915+105}

\newcommand{\cloudy}{{\sc cloudy}}
\newcommand{\xstar}{{\sc xstar}}

%%%%%%%%%%%%%%%%%%%%%%%%%%%%%%%%%%%%%%%%%%%%%%%%
\begin{document}
\title[]{}
\author[]{ }

\date{Accepted ---. Received ---; in
  original form --- }

\label{firstpage}
\pagerange{\pageref{firstpage}--\pageref{lastpage}} \pubyear{2016}
\maketitle

\begin{abstract}
\end{abstract}

\begin{keywords}
\end{keywords}

\section{Introduction}

\section{The main system of equations}

The crucial part of all the dynamic equations is the non-linear term
$(\vector{v}\nabla) \vector{v}$, containing all the inertial terms. In
spherical coordinates:
\begin{equation}\label{E:vdv}
  \begin{array}{l}
\displaystyle  (\vector{v}\nabla) \vector{v} = \left( v_r \pardir{r}{v_r} +
  \frac{v_\theta}{r} \pardir{\theta}{v_r} + \frac{v_\varphi}{r \sin
    \theta} \pardir{\varphi}{v_r} - \frac{v_\theta^2+v_\varphi^2}{r}\right)
  \vector{e}^r  \\
\displaystyle   \qquad{}   + \left( v_r \pardir{r}{v_\theta} + \frac{v_\theta}{r}\pardir{\theta}{v_\theta} +
  \frac{v_\varphi}{r\sin\theta} \pardir{\varphi}{v_\theta} + \frac{v_\theta
    v_r}{r} - \frac{v_\varphi^2}{r}\cot\theta\right) \vector{e}^\theta
   \\
\displaystyle   \qquad{}   +
   \left( v_r \pardir{r}{v_\varphi}  + 
   \frac{v_\theta}{r}\pardir{\theta}{v_\varphi} +
   \frac{v_\varphi}{r\sin\theta}\pardir{\varphi}{v_\varphi} + \frac{v_\varphi
     v_r}{r} + \frac{v_\varphi
     v_\theta}{r}\cot\theta\right) \vector{e}^\varphi,\\
   \end{array}
\end{equation}
where $\vector{e}$ are unit vectors along different coordinate directions, $r$
is radius, $\theta$ is polar angle, $\varphi$ is azimuthal
angle. Directly, this expression will be used for the vertical structure
equations. On the sphere, the velocities may be converted to
vorticity  and divergence:
\begin{equation}\label{E:vort}
  \displaystyle   \vector{\omega} = \left[ \nabla \times \vector{v}\right],
\end{equation}
\begin{equation}\label{E:div}
\displaystyle   \delta = (\nabla \cdot \vector{v}) = \frac{1}{r^2}\ppardir{r}{r^2v_r} +
  \frac{1}{r\sin \theta} \ppardir{\theta}{\sin \theta v_\theta} + \frac{1}{r\sin \theta}\pardir{\varphi}{v_\varphi} .
\end{equation}
Taking curl of Euler equation allows to directly derive the equation for
vorticity
\begin{equation}\label{E:Ecurl}
\displaystyle   \pardir{t}{\left[\nabla \vector{v}\right]} +
  (\vector{v}\nabla)\left[\nabla \vector{v}\right] = (\vector{\omega} \cdot
\nabla) \vector{v} - \vector{\omega} (\nabla \cdot \vector{v}) +
  \frac{1}{\rho^2}\left[\nabla p \times \nabla \rho \right].
\end{equation}
The right-hand-side term is important if motion is baroclinic. Baroclinic term
allows to generate vorticity from non-axisymmetric perturbations of density
and temperature, hence it is important for any realistic calculations. We will
neglect the radial velocities and consider only the radial component of
$\vector{\omega}$ that means all the motions are restricted to the surface of
the sphere and the vertical relaxation timescale is much smaller than the
global dynamical scales. Setting $\vector{\omega} = \vector{e}_r \omega$ and
substituting it to equation~(\ref{E:Ecurl}) gives
\begin{equation}\label{E:Ecurl:r}
\displaystyle  \pardir{t}{\omega} + \nabla \cdot (\omega \vector{v}) =
\frac{1}{\rho^2}\left[\nabla p \times \nabla \rho \right].
\end{equation}
Additional kinetic terms disappear because of the two-dimensional nature of
the flow, not due to incompressibility assumption that also zeros these
terms. Current version of the code adopts a fixed equation of state hence
$\nabla p \propto \nabla \rho $, and the right-hand side is zero. Note that
integration in vertical coordinate is not required in this case. 

Another equation comes from taking divergence of Euler equation. It is rather
non-trivial to expand the advection term, $(\nabla \cdot
((\vector{v}\nabla)\vector{v}))$, therefore first let us show that
\begin{equation}\label{E:vomega}
\left(\nabla \cdot \left[ \vector{v} \vector{\omega}\right] \right) = 
 \nabla^2\frac{v^2}{2} - \nabla \cdot \left((\vector{v}\nabla)\vector{v}\right).
\end{equation}
The last term here is identical to the advective left-hand-side term in Euler
equation derivative, hence
\begin{equation}\label{E:delta}
\pardir{t}{\delta} = \left(\nabla \cdot \left[ \vector{v}
  \vector{\omega}\right] \right) -  \nabla^2 B.
\end{equation}
where
\begin{equation}\label{E:Bernoulli}
B = \frac{v^2}{2} + \Phi + h
\end{equation}
is Bernoulli integral, and $h = \int \frac{1}{\rho}dp$ is enthalpy. For the
isothermal case we start with, $p =  \rho c_{\rm s}^2$, where the speed of
sound $c_{\rm s}^2$ is constant, and $h= c_{\rm s}^2 \ln\rho$. Integration
with height does not change the overall structure of the equation, as the
vertical scalehight is always identical for density and pressure. 

\subsection{Vertical balance}

\section{Spectral code}

\section{Dissipation}

To stabilize the algorithm, we use a hyperdiffusion method selective for the
highest harmonics. Dissipation operator for both quantities, vorticity and
divergence, equals
\begin{equation}
\displaystyle  D = \frac{1}{t_{D}}\left(\nabla^4 - \frac{4}{R^4}\right),
\end{equation}
where the second, correction, term serves to ensure conservation of the total
angular momentum and mass (see \citet{swater}). In the spectral domain, this
corresponds to
\begin{equation}
\displaystyle   \tilde{D} = \frac{1}{t_{D}}\left( ik^4 -
\frac{4}{R^4}\right)\simeq e^{}
\end{equation}

\section{Realistic vertical structure}

We assume the atmosphere thin, that justifies the usage of plane-parallel
approximation and constant effective gravity $g=const$. Vertical component of
momentum equation, reduced to hydrostatics due to zero vertical velocities
\alert{should be always set $v_r=0$}
\begin{equation}\label{E:vert:p}
\frac{1}{\rho}\pardir{r}{p}
=g_{\rm eff} = -\frac{GM}{r^2} + \frac{v_\theta^2+v_\varphi^2}{r}.
\end{equation}
To restore the vertical density profile, let us assume, following \citet{IS99}
and \citet{VP06}, that most of the heat is released at the bottom of the
atmosphere, and hence radiation flux $F$ is constant with height
\begin{equation}\label{E:vert:pr}
F = - \frac{c}{3\varkappa \rho} \frac{p_r}{dr},
\end{equation}
where $p_r$ is radiation pressure.
Together, equations (\ref{E:vert:p}) and (\ref{E:vert:pr}) imply constant
ratio pressure ratio $p_r / p$ as long as opacity is constant with height. 



\section{Inertial modes and sonic horizons}

\subsection{Derivation of the dispersion equation}

Using WKB method \citep{WKB}, one can linearize the set of dynamic equations
and come up with a dispersion relation for short-wavelength waves on a sphere
in presence of differential rotation. Density variations $\rho = \rho_0
+\delta \rho(\theta,\varphi, t)$, azimuthal velocity $v_\varphi = \Omega(\theta) \sin\theta +
\delta v_\varphi(\theta,\varphi, t)$, and latitudinal velocity $v_\theta = \delta
v_\theta(\theta,\varphi, t)$. Latitudinal velocity has the same (first) order
as the azimuthal velocity perturbation. All the perturbations will be
expressed in exponential form $\propto \exp(\i(\omega t - k_\theta \theta
- k_\varphi \varphi))$. \alert{Will spherical harmonics be better?}
Wavenumbers would be expressed in the units of inverse sphere radius. 

Continuity equation perturbation, written in WKB assumptions, becomes
\begin{equation}\label{E:WKB:continuity}
\displaystyle \left(\omega-k_\varphi \Omega\right) \frac{\delta \rho}{\rho} = k_\theta v_\theta +
\frac{1}{\sin \theta} k_\varphi \delta v_\varphi.
\end{equation}
The two tangential Euler equations may be in general form written, ignoring
second-order terms, as
\begin{equation}\label{E:WKB:Eulertheta}
\displaystyle \pardir{t}{v_\theta} + \frac{v_\varphi}{\sin \theta}
\pardir{\varphi}{v_\theta} - v_\varphi^2 \cot \theta = -\cs^2
\pardir{\theta}{\ln \rho}
\end{equation}
and
\begin{equation}\label{E:WKB:Eulerphi}
\displaystyle  \pardir{t}{v_\phi} + v_\theta \pardir{\theta}{v_\varphi}
  +\frac{v_\varphi}{\sin\theta}\pardir{\varphi}{v_\varphi}+ v_\varphi v_\theta
  \cot \theta = -\frac{\cs^2 }{\sin \theta} \pardir{\varphi}{\ln \rho}.
\end{equation}
In WKB approach, and after substitution the expression for the variations
of $\rho$ from (\ref{E:WKB:continuity}), the equations become, respectively,
\begin{equation}\label{E:WKB:theta}
\displaystyle \left(\tilde{\omega}^2 - \cs^2 k_\theta^2\right) v_\theta=
\left( \cs^2 \frac{k_\theta k_\varphi}{\sin\theta} - 2 \i \Omega
\tilde{\omega} \cos\theta\right) \delta v_\varphi,
\end{equation}
and
\begin{equation}\label{E:WKB:phi}
\displaystyle
\left(\tilde{\omega}\left(\tilde{\omega}-k_\varphi\Omega\right)-\frac{\cs^2
  k_\varphi^2}{\sin^2\theta}\right)\delta v_\varphi = \left( \cs^2 k_\theta
k_\varphi  + \i \tilde{\omega} \pardir{\theta}{\Omega \sin^2\theta}\right)
\frac{v_\theta}{\sin \theta},
\end{equation}
where $\tilde{\omega} = \omega - k_\varphi \Omega$. 
Excluding the velocity components yields a dispersion equation
\begin{equation}\label{E:WKB:deq}
\displaystyle \left( \tilde{\omega}^2 - \cs^2 k_\theta^2\right)\left(
\tilde{\omega}-k_\varphi \Omega\right) - \frac{\cs^2
  k_\varphi^2}{\sin^2\theta}\tilde{\omega} = \i \cs^2 k_\varphi k_\theta
\sin\theta \pardir{\theta}{\Omega} + 2\Omega \tilde{\omega} \cot\theta
\ppardir{\theta}{\Omega \sin^2\theta}.
\end{equation}
Two important specific cases may be reproduced when $\Omega \to 0$ (sonic
waves) and $\cs \to 0$ (inertial waves):
\begin{equation}\label{E:WKB:sonic}
 \displaystyle  \omega_{\rm sonic} = \cs^2 \left( k_\theta^2 + \frac{k_\varphi^2}{\sin^2\theta}\right),
\end{equation}
\begin{equation}\label{E:WKB:inertial}
\displaystyle   \omega_{\rm inertial} = \frac{3}{2}k_\varphi \Omega \pm
  \frac{1}{2}\sqrt{(k_\varphi \Omega)^2 + 4\varkappa^2},
\end{equation}
where
\begin{equation}\label{E:WKB:varkappa}
  \varkappa^2 = 2\Omega \cot\theta \ppardir{\theta}{\Omega \sin^2\theta}
\end{equation}
is the latitudinal epicyclic frequency. In the case of
$k_\theta$ and $\Omega=const$, $\omega = \varkappa = 2\Omega \cos\theta$
reproduces the Coriolis oscillation regime. Isomomentum rotation, on the other
hand, reproduces $\varkappa = 0$ and does not have oscillating purely latitudinal
modes.

\alert{Seems that the real variability patterns are due to rotation, and
  $2\Omega$ comes rather from a two-armed structure. But I have never seen
  this derivation of the oscillation  modes on a sphere, so let us keep it. }

\subsection{Sonic horizon}

\bibliographystyle{mnras}
\bibliography{mybib}

\label{lastpage}

\end{document}
