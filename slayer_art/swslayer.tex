\documentclass[usenatbib,onecolumn]{mnras}
\pdfoutput=1
\pdfminorversion=5

%\usepackage{amsmath}
\usepackage{mathtext,amssymb,amsmath}
\usepackage{epsfig}
\usepackage{graphics}
\usepackage{url}

\usepackage{times}
%\usepackage{helvet}
\usepackage[T1]{fontenc} 
\usepackage{aecompl}

%%%%% AUTHORS - PLACE YOUR OWN MACROS HERE %%%%%

\newcommand{\alert}[1]{\color{red} #1\color{black}}

\renewcommand{\vector}[1]{\ensuremath{\mathbf{#1}}}
\renewcommand{\div}{\ensuremath{-}}

\newcommand{\bea}{\begin{eqnarray}}
\newcommand{\eea}{\end{eqnarray}}
  
\newcommand{\Mach}{\ensuremath{\mathcal{M}}}
\newcommand{\rot}{\ensuremath{\mathbf{curl\,}}}
\newcommand{\mdot}{\ensuremath{\dot{m}}}
\newcommand{\Msun}{\ensuremath{\,\rm M_\odot}}
\newcommand{\Msunyr}{\ensuremath{\,\rm M_\odot\, \rm yr^{-1}}}
\newcommand{\ergl}{\ensuremath{\,\rm erg\, s^{-1}}}
\newcommand{\Gyr}{\ensuremath{\,\rm Gyr}}
\newcommand{\yr}{\ensuremath{\,\rm yr}}
\newcommand{\mum}{\ensuremath{\,\rm \mu m}}
\newcommand{\pc}{\ensuremath{\,\rm pc}}
\newcommand{\cmc}{\ensuremath{\,\rm cm^{-3}}}
\newcommand{\cmsq}{\ensuremath{\,\rm cm^{-2}}}

\newcommand{\Ry}{\ensuremath{\,{\rm Ry}}}
\newcommand{\eV}{\ensuremath{\,{\rm eV}}}
\newcommand{\keV}{\ensuremath{\,{\rm keV}}}
\newcommand{\GeV}{\ensuremath{\,{\rm GeV}}}
\newcommand{\TeV}{\ensuremath{\,{\rm TeV}}}
\newcommand{\AAA}{\ensuremath{\,{\rm \AA}}}
% \newcommand{\ion}[2]{ \ensuremath{ \rm #1\,\sc #2 }}

\newcommand{\gfun}[1]{\ensuremath{{\rm G}\,(#1)}}
\newcommand{\acos}{\ensuremath{\rm acos}\,}
\newcommand{\litwo}{\ensuremath{\rm Li}_2\,}
\newcommand{\lithree}{\ensuremath{\rm Li}_3\,}
\newcommand{\li}[2]{{\rm Li}_{#1}\!\left(#2\right)}
\newcommand{\gf}{\ensuremath{\frac{\sqrt{g_{\varphi\varphi}}}{\alpha}}}
\newcommand{\pardir}[2]{\ensuremath{\frac{\partial #2}{\partial #1} }}
\newcommand{\pardirsec}[2]{\ensuremath{\frac{\partial^2 #2}{\partial #1^2} }}
\newcommand{\ppardir}[2]{\ensuremath{\frac{\partial }{\partial #1} \left( #2\right)}}
\newcommand{\eps}{\epsilon}

%\newcommand{\tcfour}{3C\,454.3}
%\newcommand{\tctwo}{3C\,279}
%\newcommand{\pkst}{PKS 1222$+$21}
%\newcommand{\pksf}{PKS 1510$-$89}
% \newcommand{\grs}{GRS~1915+105}

\newcommand{\cloudy}{{\sc cloudy}}
\newcommand{\xstar}{{\sc xstar}}

%%%%%%%%%%%%%%%%%%%%%%%%%%%%%%%%%%%%%%%%%%%%%%%%
\begin{document}
\title[]{}
\author[]{ }

\date{Accepted ---. Received ---; in
  original form --- }

\label{firstpage}
\pagerange{\pageref{firstpage}--\pageref{lastpage}} \pubyear{2016}
\maketitle

\begin{abstract}
\end{abstract}

\begin{keywords}
\end{keywords}

\section{Introduction}

\section{The main system of equations}

The crucial part of all the dynamic equations is the non-linear term
$(\vector{v}\nabla) \vector{v}$, containing all the inertial terms. In
spherical coordinates:
\begin{equation}\label{E:vdv}
  \begin{array}{l}
\displaystyle  (\vector{v}\nabla) \vector{v} = \left( v_r \pardir{r}{v_r} +
  \frac{v_\theta}{r} \pardir{\theta}{v_r} + \frac{v_\varphi}{r \sin
    \theta} \pardir{\varphi}{v_r} - \frac{v_\theta^2+v_\varphi^2}{r}\right)
  \vector{e}^r  \\
\displaystyle   \qquad{}   + \left( v_r \pardir{r}{v_\theta} + \frac{v_\theta}{r}\pardir{\theta}{v_\theta} +
  \frac{v_\varphi}{r\sin\theta} \pardir{\varphi}{v_\theta} + \frac{v_\theta
    v_r}{r} - \frac{v_\varphi^2}{r}\cot^2\theta\right) \vector{e}^\theta
   \\
\displaystyle   \qquad{}   +
   \left( v_r \pardir{r}{v_\varphi}  + 
   \frac{v_\theta}{r}\pardir{\theta}{v_\varphi} +
   \frac{v_\varphi}{r\sin\theta}\pardir{v_\varphi}{\varphi} + \frac{v_\varphi
     v_r}{r}\right) \vector{e}^\varphi,\\
   \end{array}
\end{equation}
where $\vector{e}$ are unit vectors along different coordinate directions, $r$
is radius, $\theta$ is polar angle, $\varphi$ is azimuthal
angle. Directly, this expression will be used for the vertical structure
equations. On the sphere, the velocities may be converted to
vorticity  and divergence:
\begin{equation}\label{E:vort}
  \displaystyle   \vector{\omega} = \left[ \nabla \times \vector{v}\right],
\end{equation}
\begin{equation}\label{E:div}
\displaystyle   \delta = (\nabla \cdot \vector{v}) = \frac{1}{r^2}\ppardir{r}{r^2v_r} +
  \frac{1}{r\sin \theta} \ppardir{\theta}{\sin \theta v_\theta} + \frac{1}{r\sin \theta}\pardir{\varphi}{v_\varphi} .
\end{equation}
Taking curl of Euler equation allows to directly derive the equation for
vorticity
\begin{equation}\label{E:Ecurl}
\displaystyle   \pardir{t}{\left[\nabla \vector{v}\right]} +
  (\vector{v}\nabla)\left[\nabla \vector{v}\right] = (\vector{\omega} \cdot
\nabla) \vector{v} - \vector{\omega} (\nabla \cdot \vector{v}) +
  \frac{1}{\rho^2}\left[\nabla p \times \nabla \rho \right].
\end{equation}
The right-hand-side term is important if motion is baroclinic. Baroclinic term
allows to generate vorticity from non-axisymmetric perturbations of density
and temperature, hence it is important for any realistic calculations. We will
neglect the radial velocities and consider only the radial component of
$\vector{\omega}$ that means all the motions are restricted to the surface of
the sphere and the vertical relaxation timescale is much smaller than the
global dynamical scales. Setting $\vector{\omega} = \vector{e}_r \omega$ and
substituting it to equation~(\ref{E:Ecurl}) gives
\begin{equation}\label{E:Ecurl:r}
\displaystyle  \pardir{t}{\omega} + \nabla \cdot (\omega \vector{v}) =
\frac{1}{\rho^2}\left[\nabla p \times \nabla \rho \right].
\end{equation}
Additional kinetic terms disappear because of the two-dimensional nature of
the flow, not due to incompressibility assumption that also zeros these
terms. Current version of the code adopts a fixed equation of state hence
$\nabla p \propto \nabla \rho $, and the right-hand side is zero. Note that
integration in vertical coordinate is not required in this case. 

Another equation comes from taking divergence of Euler equation. It is rather
non-trivial to expand the advection term, $(\nabla \cdot
((\vector{v}\nabla)\vector{v}))$, therefore first let us show that
\begin{equation}\label{E:vomega}
\left(\nabla \cdot \left[ \vector{v} \vector{\omega}\right] \right) = 
 \nabla^2\frac{v^2}{2} - \nabla \cdot \left((\vector{v}\nabla)\vector{v}\right).
\end{equation}
The last term here is identical to the advective left-hand-side term in Euler
equation derivative, hence
\begin{equation}\label{E:delta}
\pardir{t}{\delta} = \left(\nabla \cdot \left[ \vector{v}
  \vector{\omega}\right] \right) -  \nabla^2 B.
\end{equation}
where
\begin{equation}\label{E:Bernoulli}
B = \frac{v^2}{2} + \Phi + h
\end{equation}
is Bernoulli integral, and $h = \int \frac{1}{\rho}dp$ is enthalpy. For the
isothermal case we start with, $p =  \rho c_{\rm s}^2$, where the speed of
sound $c_{\rm s}^2$ is constant, and $h= c_{\rm s}^2 \ln\rho$. Integration
with height does not change the overall structure of the equation, as the
vertical scalehight is always identical for density and pressure. 

\section{Vertical balance}



\bibliographystyle{mnras}
\bibliography{mybib}

\label{lastpage}

\end{document}
