\documentclass[usenatbib,onecolumn]{mnras}
\pdfoutput=1
\pdfminorversion=5

%\usepackage{amsmath}
\usepackage{mathtext,amssymb,amsmath}
\usepackage{epsfig}
\usepackage{graphics}
\usepackage{url}

\usepackage{times}
%\usepackage{helvet}
\usepackage[T1]{fontenc} 
\usepackage{aecompl}

%%%%% AUTHORS - PLACE YOUR OWN MACROS HERE %%%%%

\newcommand{\alert}[1]{\color{red} #1\color{black}}

\renewcommand{\vector}[1]{\ensuremath{\mathbf{#1}}}
\renewcommand{\div}{\ensuremath{-}}

\newcommand{\bea}{\begin{eqnarray}}
\newcommand{\eea}{\end{eqnarray}}
  
\newcommand{\Mach}{\ensuremath{\mathcal{M}}}
\newcommand{\rot}{\ensuremath{\mathbf{curl\,}}}
\newcommand{\mdot}{\ensuremath{\dot{m}}}
\newcommand{\Msun}{\ensuremath{\,\rm M_\odot}}
\newcommand{\Msunyr}{\ensuremath{\,\rm M_\odot\, \rm yr^{-1}}}
\newcommand{\ergl}{\ensuremath{\,\rm erg\, s^{-1}}}
\newcommand{\Gyr}{\ensuremath{\,\rm Gyr}}
\newcommand{\yr}{\ensuremath{\,\rm yr}}
\newcommand{\mum}{\ensuremath{\,\rm \mu m}}
\newcommand{\pc}{\ensuremath{\,\rm pc}}
\newcommand{\cmc}{\ensuremath{\,\rm cm^{-3}}}
\newcommand{\cmsq}{\ensuremath{\,\rm cm^{-2}}}

\newcommand{\Ry}{\ensuremath{\,{\rm Ry}}}
\newcommand{\eV}{\ensuremath{\,{\rm eV}}}
\newcommand{\keV}{\ensuremath{\,{\rm keV}}}
\newcommand{\GeV}{\ensuremath{\,{\rm GeV}}}
\newcommand{\TeV}{\ensuremath{\,{\rm TeV}}}
\newcommand{\AAA}{\ensuremath{\,{\rm \AA}}}
% \newcommand{\ion}[2]{ \ensuremath{ \rm #1\,\sc #2 }}

\newcommand{\gfun}[1]{\ensuremath{{\rm G}\,(#1)}}
\newcommand{\acos}{\ensuremath{\rm acos}\,}
\newcommand{\litwo}{\ensuremath{\rm Li}_2\,}
\newcommand{\lithree}{\ensuremath{\rm Li}_3\,}
\newcommand{\li}[2]{{\rm Li}_{#1}\!\left(#2\right)}
\newcommand{\gf}{\ensuremath{\frac{\sqrt{g_{\varphi\varphi}}}{\alpha}}}
\newcommand{\pardir}[2]{\ensuremath{\frac{\partial #2}{\partial #1} }}
\newcommand{\pardirsec}[2]{\ensuremath{\frac{\partial^2 #2}{\partial #1^2} }}
\newcommand{\ppardir}[2]{\ensuremath{\frac{\partial }{\partial #1} \left( #2\right)}}
\newcommand{\eps}{\epsilon}
\newcommand{\cs}{\ensuremath{c_{\rm s}}}
\renewcommand{\i}{\ensuremath{{\rm i}}}

%%%%%%%%%%%%%%%%%%%%%%%%%%%%%%%%%%%%%%%%%%%%%%%%
\begin{document}
\title[]{}
\author[]{ }

\date{Accepted ---. Received ---; in
  original form --- }

\label{firstpage}
\pagerange{\pageref{firstpage}--\pageref{lastpage}} \pubyear{2016}
\maketitle

\begin{abstract}
\end{abstract}

\begin{keywords}
\end{keywords}

\section{Introduction}

\section{Equation set for a spreading layer}

\subsection{Kinematics on the sphere}\label{E:kinema}

The crucial part of all the dynamic equations is the non-linear term
$(\vector{v}\nabla) \vector{v}$, containing all the inertial terms. In
spherical coordinates:
\begin{equation}\label{E:vdv}
  \begin{array}{l}
\displaystyle  (\vector{v}\nabla) \vector{v} = \left( v_r \pardir{r}{v_r} +
  \frac{v_\theta}{r} \pardir{\theta}{v_r} + \frac{v_\varphi}{r \sin
    \theta} \pardir{\varphi}{v_r} - \frac{v_\theta^2+v_\varphi^2}{r}\right)
  \vector{e}^r  \\
\displaystyle   \qquad{}   + \left( v_r \pardir{r}{v_\theta} + \frac{v_\theta}{r}\pardir{\theta}{v_\theta} +
  \frac{v_\varphi}{r\sin\theta} \pardir{\varphi}{v_\theta} + \frac{v_\theta
    v_r}{r} - \frac{v_\varphi^2}{r}\cot\theta\right) \vector{e}^\theta
   \\
\displaystyle   \qquad{}   +
   \left( v_r \pardir{r}{v_\varphi}  + 
   \frac{v_\theta}{r}\pardir{\theta}{v_\varphi} +
   \frac{v_\varphi}{r\sin\theta}\pardir{\varphi}{v_\varphi} + \frac{v_\varphi
     v_r}{r} + \frac{v_\varphi
     v_\theta}{r}\cot\theta\right) \vector{e}^\varphi,\\
   \end{array}
\end{equation}
where $\vector{e}$ are unit vectors along different coordinate directions, $r$
is radius, $\theta$ is polar angle, $\varphi$ is azimuthal
angle. Directly, this expression will be used for the vertical structure
equations. On the sphere, the velocities may be converted to
vorticity  and divergence:
\begin{equation}\label{E:vort}
  \displaystyle   \vector{\omega} = \left[ \nabla \times \vector{v}\right],
\end{equation}
\begin{equation}\label{E:div}
\displaystyle   \delta = (\nabla \cdot \vector{v}) = \frac{1}{r^2}\ppardir{r}{r^2v_r} +
  \frac{1}{r\sin \theta} \ppardir{\theta}{\sin \theta v_\theta} + \frac{1}{r\sin \theta}\pardir{\varphi}{v_\varphi} .
\end{equation}
Taking curl of Euler equation allows to directly derive the equation for
vorticity
\begin{equation}\label{E:Ecurl}
\displaystyle   \pardir{t}{\left[\nabla \vector{v}\right]} +
  (\vector{v}\nabla)\left[\nabla \vector{v}\right] = (\vector{\omega} \cdot
\nabla) \vector{v} - \vector{\omega} (\nabla \cdot \vector{v}) +
  \frac{1}{\rho^2}\left[\nabla p \times \nabla \rho \right].
\end{equation}
The right-hand-side term is important if motion is baroclinic. Baroclinic term
allows to generate vorticity from non-axisymmetric perturbations of density
and temperature, hence it is important for any realistic calculations. We will
neglect the radial velocities and consider only the radial component of
$\vector{\omega}$ that means all the motions are restricted to the surface of
the sphere and the vertical relaxation timescale is much smaller than the
global dynamical scales. Setting $\vector{\omega} = \vector{e}_r \omega$ and
substituting it to equation~(\ref{E:Ecurl}) gives
\begin{equation}\label{E:Ecurl:r}
\displaystyle  \pardir{t}{\omega} + \nabla \cdot (\omega \vector{v}) =
\frac{1}{\rho^2}\left[\nabla p \times \nabla \rho \right].
\end{equation}
Additional kinetic terms disappear because of the two-dimensional nature of
the flow, not due to incompressibility assumption that also zeros these
terms. Current version of the code adopts a fixed equation of state hence
$\nabla p \propto \nabla \rho $, and the right-hand side is zero. Note that
integration in vertical coordinate is not required in this case. 

Another equation comes from taking divergence of Euler equation. It is rather
non-trivial to expand the advection term, $(\nabla \cdot
((\vector{v}\nabla)\vector{v}))$, therefore first let us show that
\begin{equation}\label{E:vomega}
\left(\nabla \cdot \left[ \vector{v} \vector{\omega}\right] \right) = 
 \nabla^2\frac{v^2}{2} - \nabla \cdot \left((\vector{v}\nabla)\vector{v}\right).
\end{equation}
The last term here is identical to the advective left-hand-side term in Euler
equation derivative, hence
\begin{equation}\label{E:delta}
\pardir{t}{\delta} = \left(\nabla \cdot \left[ \vector{v}
  \vector{\omega}\right] \right) -  \nabla^2 B.
\end{equation}
where
\begin{equation}\label{E:Bernoulli}
B = \frac{v^2}{2} + \Phi + h
\end{equation}
is Bernoulli integral, and $h = \int \frac{1}{\rho}dp$ is enthalpy. For the
isothermal case we start with, $p =  \rho c_{\rm s}^2$, where the speed of
sound $c_{\rm s}^2$ is constant, and $h= c_{\rm s}^2 \ln\rho$. Integration
with height does not change the overall structure of the equation, as the
vertical scalehight is always identical for density and pressure. 

\subsection{Dissipation}

To stabilize the algorithm, we use a diffusion method selective for the
highest harmonics [ref]. Dissipation operator for vorticity and
divergence, equals
\begin{equation}
\displaystyle  D = \frac{\Delta t}{t_{\rm D}} \frac{1}{k_{\rm min}^2}\nabla^2 ,
\end{equation}
where $t_{\rm D}$ has the physical meaning of diffusion time scale, and
$\Delta t$ is the time step. 

\subsection{Realistic vertical structure}

\alert{re-write this section!}

We assume the atmosphere thin, that justifies the usage of plane-parallel
approximation and constant effective gravity $g=const$. Vertical component of
momentum equation, reduced to hydrostatics due to zero vertical velocities
\alert{should be always set $v_r=0$?}
\begin{equation}\label{E:vert:p}
\frac{1}{\rho}\pardir{r}{p}
=g_{\rm eff} = -\frac{GM}{r^2} + \frac{v_\theta^2+v_\varphi^2}{r}.
\end{equation}
To restore the vertical density profile, let us assume, following \citet{IS99}
and \citet{VP06}, that most of the heat is released at the bottom of the
atmosphere, and hence radiation flux $F$ is constant with height
\begin{equation}\label{E:vert:pr}
F = - \frac{c}{3\varkappa \rho} \frac{dp_r}{dr},
\end{equation}
where $p_r$ is radiation pressure.
Together, equations (\ref{E:vert:p}) and (\ref{E:vert:pr}) imply constant
ratio pressure ratio $p_r / p$ as long as opacity is constant with height. 
Hence, gas, radiation, and total pressure scale with each other, and the gas-to-total
pressure ratio equals
\begin{equation}\label{E:vert:beta}
\beta = 1- \frac{\varkappa F }{cg_{\rm eff}}.
\end{equation}
Proportionality of pressures also implies $p \propto \rho T \propto T^4$, that
leads to $p\propto \rho^{4/3}$, an effectively polytropic law. Integration of
equation~(\ref{E:vert:p}) yields
\begin{equation}\label{E:vert:prho}
\frac{p}{\rho} = \frac{3}{4} g_{\rm eff} \left( r_{\rm surface} - r \right).
\end{equation}
Vertical density profile is $\rho \propto (r_{\rm surface} - r)^{3}$. This
allows to connect the vertically integrated quantities,
\begin{equation}\label{E:vert:pisigma}
\Pi = \frac{4}{7} \frac{p_0}{\rho_0} \Sigma,
\end{equation}
where
\begin{equation}\label{E:vert:p0rho0}
\frac{p_0}{\rho_0} = \frac{1}{\beta} \frac{k}{m} \left(\frac{3}{4}\varkappa
\Sigma \frac{F}{\sigma_{\rm SB}} \right)^{1/4},
\end{equation}
where $F$ is the local energy flux assumed to be released somewhere on the
bottom of the atmosphere.

In a more general case, the flux emitted from the surface is not equal to the
energy generated inside the layer, but one can fix the vertical scalings of
the thermodynamical quantities while allowing the total energy content to
change in time. This assumption set leads to the following implicit equation
for $\beta$:
\begin{equation}\label{E:vert:beta:new}
\frac{\beta}{\left(1-\beta\right)^{1/4}} = \frac{4}{7} \frac{k}{m}
\left(\frac{3}{4}\frac{c}{\sigma_{\rm SB}} g_{\rm eff} \Sigma \right)^{1/4}
\frac{\Sigma}{\Pi} \simeq 2\times 10^{-6} \mu^{-1} \frac{\Sigma c^2}{\Pi}
\left(g_{\rm eff}^*\right)^{1/4} M_1^{1/4} \Sigma_1^{1/4},
\end{equation}
and dimensionless gravity
\begin{equation}\label{E:vert:gravity}
g_{\rm eff}^* = \frac{1}{r^2} - \frac{v_\varphi^2 + v_\theta^2}{r}.
\end{equation}

\subsection{Energy conservation}

In general form, energy conservation implies \citep{SP06}:
\begin{equation}\label{E:energy:general}
\ppardir{t}{\frac{1}{2}\rho v^2 + \varepsilon}+\nabla \cdot \left(
\left(\frac{1}{2}\rho v^2 + \varepsilon + p \right)
\vector{v}\right) = q_{\rm NS} - q^{-},
\end{equation}
where the right hand side accounts for heat exchange with the neutron star ($q_{\rm NS}$) and radiation losses from the surface
$q^{-}$. After integration, all the $q$ quantities will result in
corresponding $Q$ quantities: fluxes through the surface and energy release
per unit area.
Diffusive heat transport by conduction or radiation is ignored here,
as they are included in vertical balance and transfer.
From Euler equation, by multiplying it by $\vector{v}$,
\begin{equation}\label{E:energy:Euler}
  \ppardir{t}{\frac{1}{2}\rho v^2}+\nabla \cdot \left(\frac{1}{2}\rho v^2
  \vector{v}\right) = -(\vector{v} \cdot \nabla) p - q^+,
\end{equation}
where $q^+$ is viscous dissipation. Subtracting (\ref{E:energy:Euler}) from
(\ref{E:energy:general}) yields
\begin{equation}\label{E:energy:thermal}
\pardir{t}{\varepsilon} + \nabla \cdot \left( \varepsilon \vector{v}\right) =
p \delta + q^+ + q_{\rm NS} - q^-.
\end{equation}
Internal energy density $\varepsilon$ consists of gas energy density
$\varepsilon_{\rm g} = \frac{3}{2} p_{\rm g}$ and radiation energy density
$\varepsilon_{\rm r} = 3 p_{\rm r}$, that implies $\varepsilon = 3\left(
1-\frac{\beta}{2}\right) p$. As $\beta$ is constant with height, integration
gives
\begin{equation}\label{E:energy:Pi}
\pardir{t}{\Pi} + \nabla \cdot \left( \Pi \vector{v}\right) = \frac{1}{3\left(
  1-\frac{\beta}{2}\right)} \left(\delta \Pi + Q^+ - Q^- + Q_{\rm NS}\right),
\end{equation}
where $Q^+$ is the heat released in the spreading layer, $Q_{\rm NS}$ is the
heat received from the neutron star (may be negative, if the neutron star is
cooler), and $Q^-$ is the radiation flux lost from the surface.

Vertically integrated pressure may be treated as an independent quantity,
connected

Energy release
\begin{equation}\label{E:energy:qplus}
Q^+ = \Sigma \vector{v} \cdot \left.\frac{d\vector{v}}{dt}\right|_{\rm dissipation},
\end{equation}
\begin{equation}\label{E:energy:qminus}
Q^- = \frac{7}{3} \frac{c}{\varkappa \Sigma}(1-\beta) \Pi,
\end{equation}
and $Q_{\rm NS}$ will be assumed zero so far. Pressure ratio $\beta$ is
calculated according to equation (\ref{E:vert:beta:new}).

\subsection{Complete set of equations}

Spatial and temporal scales are set by the free fall,
\begin{equation}\label{E:set:timescale}
  t_{\rm ff} = \frac{GM}{c^3},
\end{equation}
\begin{equation}\label{E:set:radius}
  R_{\rm g} = \frac{GM}{c^2},
\end{equation}
hence, all the kinematical quantities are naturally normalized by combinations
of these quantities. There is no natural convenient scale for $\Sigma$, as the
opacity does not play as fundamental a role, hence we normalize $\Sigma$ by
$1{\rm cm^2 g^{-1}}$.

Basic dynamic equations are continuity equation
\begin{equation}\label{E:set:sigma}
\pardir{t}{\Sigma} = - \nabla \cdot (\Sigma \vector{v}) + S^+ - S^-,
\end{equation}
where the source term is set explicitly as
\begin{equation}\label{E:set:source}
S^+ = S^+_{\rm norm}e^{-(\cos\alpha/\cos\alpha_0)^2/2},
\end{equation}
and $\alpha$ is the angular distance from the direction of adopted disc
rotation axis, $\cos \alpha = \cos \theta \cos i+\sin \theta \sin i \cos
\Delta \varphi$, where $\theta$ is polar angle (co-latitude), $i$ and $\Delta
\varphi$ set the coordinates of the rotation axis of the source. Normalization
$S^+_{\rm norm}$ is proportional to mass accretion rate, $S^+_{\rm norm} \simeq
\frac{\dot{M}}{(2\pi)^{3/2} \cos\alpha_0 R_{\rm NS}^2}$ (precise to the
accuracy of $O(\alpha_0)^2$).

Sink $S^-$ is determined by the surface density only,
\begin{equation}\label{E:set:sink}
  S^- = \frac{\Sigma}{t_{\rm ff}} e^{-\Sigma_{\rm max}/\Sigma},
\end{equation}
where $\Sigma_{\rm max}$ sets the limiting value of surface density, when the
atmosphere of the NS starts precipitating at dynamical timescales.

Velocity field evolution is reasonable to trace using divergence $\delta$ and
radial vorticity $\omega$ (see section~\ref{E:kinema}). Divergence
\begin{equation}\label{E:set:div}
\pardir{t}{\delta} = \nabla \cdot [\vector{v} \times  \omega \vector{e}^r] -
\nabla^2 \left(\frac{v^2}{2}\right) - \nabla \cdot \left( \frac{1}{\Sigma}
\nabla \Pi \right) + D\delta,
\end{equation}
and vorticity
\begin{equation}\label{E:set:vort}
\pardir{t}{\omega} = -\nabla \cdot (\omega \vector{v}) -
\frac{7}{8}\nabla \times \frac{\nabla \Pi}{\Sigma}+
\frac{S^+}{\Sigma} \omega_{\rm d} + D\omega,
\end{equation}
where the last two terms account for the initial vorticity of the injected
matter, $\omega_{\rm d} \simeq 2\Omega_{\rm K}(R_{\rm NS})$. \alert{Friction
  term required!}
Energy equation may be re-written as an evolution for vertically integrated
pressure
\begin{equation}\label{E:set:press}
\pardir{t}{\Pi} + \nabla \cdot \left( \Pi \vector{v}\right) = \frac{1}{3\left(
  1-\frac{\beta}{2}\right)} \left(\delta \Pi + Q^+ - Q^- + Q_{\rm NS}\right),
\end{equation}
where 
\begin{equation}\label{E:set:qplus}
Q^+ = \Sigma \vector{v} \cdot \left.\frac{d\vector{v}}{dt}\right|_{\rm dissipation},
\end{equation}
\begin{equation}\label{E:set:qminus}
Q^- = \frac{3}{4} \frac{c}{\varkappa \Sigma}(1-\beta) p_{\rm bottom} =
\frac{3}{4} \frac{c}{\varkappa} (1-\beta) |-g_{\rm eff}|,
\end{equation}
and $Q_{\rm NS} =  \sigma_{\rm SB} T_{\rm NS}^4$, where $\beta$ is found
implicitly as
\begin{equation}\label{E:set:beta}
\frac{\beta}{\left(1-\beta\right)^{1/4}} = \frac{4}{7} \frac{k}{m}
\left(\frac{3}{4}\frac{c}{\sigma_{\rm SB} g_{\rm eff} \Sigma} \right)^{1/4}
\frac{\Sigma}{\Pi} \simeq 2\times 10^{-6} \mu^{-1} \frac{\Sigma c^2}{\Pi}
\left(g_{\rm eff}^*\right)^{1/4} M_1^{1/4} \Sigma_1^{1/4},
\end{equation}
and 
\begin{equation}\label{E:set:gravity}
g_{\rm eff}^* = \frac{1}{r^2} - \frac{v_\varphi^2 + v_\theta^2}{r}.
\end{equation}

The problem solves a set of four differential equations for four variables,
$\delta$, $\omega$, $\Sigma$, and $\Pi$.

\subsection{Initial conditions}

As an initial condition, it is reasonable to take a thin isothermal
atmosphere with the temperature set by neutron star surface effective temperature
$T_{\rm NS}$. Rotation profile is assumed to be rigid-body that sets vorticity
to $2\Omega_{\rm NS}\cos \theta$. We also assume no (or practically no)
latitudinal motions are present in the beginning.
In these assumptions, steady-state Euler equation in
latitudinal direction becomes
\begin{equation}\label{E:IC:Euler}
\Omega^2 R \cot \theta = \frac{1}{R} \ppardir{\theta}{\cs^2 \ln\Sigma},
\end{equation}
or, solving for surface density as a function of polar angle,
\begin{equation}
\left( \frac{\Omega R}{\cs}\right)^2 d\ln \sin\theta = d\ln \Sigma ,
\end{equation}
and
\begin{equation}\label{E:IC:Sigma}
\displaystyle \Sigma_{\rm init} = \Sigma_0 \left(\sin \theta \right)^{\left(\Omega R / \cs\right)^2}.
\end{equation}
Initial pressure map should conform to that of density up to the multiplier of
$\cs^2$.

As the motions are limited to pure rotation, divergence is strictly zero. To
the basic initial condition set, a small perturbation may be added ... 

\section{Inertial modes and sonic horizons}

\subsection{Derivation of the dispersion equation}

Using WKB method \citep{WKB}, one can linearize the set of dynamic equations
and come up with a dispersion relation for short-wavelength waves on a sphere
in presence of differential rotation. Density variations $\rho = \rho_0
+\delta \rho(\theta,\varphi, t)$, azimuthal velocity $v_\varphi = \Omega(\theta) \sin\theta +
\delta v_\varphi(\theta,\varphi, t)$, and latitudinal velocity $v_\theta = \delta
v_\theta(\theta,\varphi, t)$. Latitudinal velocity has the same (first) order
as the azimuthal velocity perturbation. All the perturbations will be
expressed in exponential form $\propto \exp(\i(\omega t - k_\theta \theta
- k_\varphi \varphi))$. \alert{Will spherical harmonics be better?}
Wavenumbers would be expressed in the units of inverse sphere radius. 

Continuity equation perturbation, written in WKB assumptions, becomes
\begin{equation}\label{E:WKB:continuity}
\displaystyle \left(\omega-k_\varphi \Omega\right) \frac{\delta \rho}{\rho} = k_\theta v_\theta +
\frac{1}{\sin \theta} k_\varphi \delta v_\varphi.
\end{equation}
The two tangential Euler equations may be in general form written, ignoring
second-order terms, as
\begin{equation}\label{E:WKB:Eulertheta}
\displaystyle \pardir{t}{v_\theta} + \frac{v_\varphi}{\sin \theta}
\pardir{\varphi}{v_\theta} - v_\varphi^2 \cot \theta = -\cs^2
\pardir{\theta}{\ln \rho}
\end{equation}
and
\begin{equation}\label{E:WKB:Eulerphi}
\displaystyle  \pardir{t}{v_\phi} + v_\theta \pardir{\theta}{v_\varphi}
  +\frac{v_\varphi}{\sin\theta}\pardir{\varphi}{v_\varphi}+ v_\varphi v_\theta
  \cot \theta = -\frac{\cs^2 }{\sin \theta} \pardir{\varphi}{\ln \rho}.
\end{equation}
In WKB approach, and after substitution the expression for the variations
of $\rho$ from (\ref{E:WKB:continuity}), the equations become, respectively,
\begin{equation}\label{E:WKB:theta}
\displaystyle \left(\tilde{\omega}^2 - \cs^2 k_\theta^2\right) v_\theta=
\left( \cs^2 \frac{k_\theta k_\varphi}{\sin\theta} - 2 \i \Omega
\tilde{\omega} \cos\theta\right) \delta v_\varphi,
\end{equation}
and
\begin{equation}\label{E:WKB:phi}
\displaystyle
\left(\tilde{\omega}\left(\tilde{\omega}-k_\varphi\Omega\right)-\frac{\cs^2
  k_\varphi^2}{\sin^2\theta}\right)\delta v_\varphi = \left( \cs^2 k_\theta
k_\varphi  + \i \tilde{\omega} \pardir{\theta}{\Omega \sin^2\theta}\right)
\frac{v_\theta}{\sin \theta},
\end{equation}
where $\tilde{\omega} = \omega - k_\varphi \Omega$. 
Excluding the velocity components yields a dispersion equation
\begin{equation}\label{E:WKB:deq}
\displaystyle \left( \tilde{\omega}^2 - \cs^2 k_\theta^2\right)\left(
\tilde{\omega}-k_\varphi \Omega\right) - \frac{\cs^2
  k_\varphi^2}{\sin^2\theta}\tilde{\omega} = \i \cs^2 k_\varphi k_\theta
\sin\theta \pardir{\theta}{\Omega} + 2\Omega \tilde{\omega} \cot\theta
\ppardir{\theta}{\Omega \sin^2\theta}.
\end{equation}
Two important specific cases may be reproduced when $\Omega \to 0$ (sonic
waves) and $\cs \to 0$ (inertial waves):
\begin{equation}\label{E:WKB:sonic}
 \displaystyle  \omega_{\rm sonic} = \cs^2 \left( k_\theta^2 + \frac{k_\varphi^2}{\sin^2\theta}\right),
\end{equation}
\begin{equation}\label{E:WKB:inertial}
\displaystyle   \omega_{\rm inertial} = \frac{3}{2}k_\varphi \Omega \pm
  \frac{1}{2}\sqrt{(k_\varphi \Omega)^2 + 4\varkappa^2},
\end{equation}
where
\begin{equation}\label{E:WKB:varkappa}
  \varkappa^2 = 2\Omega \cot\theta \ppardir{\theta}{\Omega \sin^2\theta}
\end{equation}
is the latitudinal epicyclic frequency. In the case of
$k_\theta$ and $\Omega=const$, $\omega = \varkappa = 2\Omega \cos\theta$
reproduces the Coriolis oscillation regime. Isomomentum rotation, on the other
hand, reproduces $\varkappa = 0$ and does not have oscillating purely latitudinal
modes.

\alert{Seems that the real variability patterns are due to rotation, and
  $2\Omega$ comes rather from a two-armed structure. But I have never seen
  this derivation of the oscillation  modes on a sphere, so let us keep
  it. And, by the way, why a two-armed structure always emerges?}

\subsection{Sonic horizon}

\bibliographystyle{mnras}
\bibliography{mybib}

\label{lastpage}

\end{document}
